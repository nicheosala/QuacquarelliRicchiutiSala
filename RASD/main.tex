\documentclass{article}
\usepackage[utf8]{inputenc}
\usepackage{graphicx}
\usepackage[super,negative]{nth}
\usepackage{fancyhdr}
\usepackage{float}
\usepackage{color, soulutf8}

\renewcommand{\thefigure}{\arabic{section}.\arabic{figure}}
\newcommand\goal[1]{\item[{[G#1]}] }
\newcommand\requirement[1]{\item[{[R#1]}] }
\newcommand\assumption[1]{\item[{[A#1]}] }
\newcommand\usecase[1]{ [UC#1] }

\begin{document}
	\begin{titlepage}
		
		\centering
		\vspace*{0.7 cm}
		\includegraphics[scale = 0.7]{images/PolimiLogo.png}\\[1 cm]
		\textsc{\large Dipartimento di Elettronica, Informazione e Bioingegneria}\\[2 cm]
		
		\rule{\linewidth}{0.2 mm} \\[0.5 cm]
		{\huge \bfseries Requirement Analysis and Specification Document (RASD)}\\
		\rule{\linewidth}{0.2 mm} \\[1.5 cm]
		
		\textsc{\Large SafeStreets}\\[0.5 cm]
		\textsc{\large - v1.0 -}\\[1 cm]
		
		\begin{minipage}{\textwidth}
			\begin{flushleft} \large
				\emph{Authors:}\\
				\textbf{Quacquarelli} Sebastiano \hfill 945071 \\
				\textbf{Ricchiuti} Simone \hfill 945613  \\
				\textbf{Sala} Nicolò \hfill 945898  \\[2 cm]
			\end{flushleft}
		\end{minipage}\\[2 cm]
		
		{\large November \nth{10} , 2019}\\[2 cm]
		
	\end{titlepage}
	
	\pagenumbering{roman}
	\tableofcontents
	
	\newpage
	\pagenumbering{arabic}
	\setcounter{page}{1}
	
	\section{Introduction}
	
		\subsection{Purpose}
		Nowadays road traffic has quickly increased. Consequently, it is more and more common to face traffic violations.\\
		SafeStreets is a system that wants to help solving this critical problem: it allows its users to notify possible traffic violations to authorities, which can evaluate them and eventually generate traffic tickets.\\
		Users are encouraged to report a violation because there is not a direct connection between the reporter and authorities: every report is anonymous.\\
		The only thing a user needs is an electronic device with an internet connection and the SafeStreets app installed. He is only asked to take a picture and select the kind of violation he wants to notify.\\
		When these two simple operations are correctly done, SafeStreets handles the report, publishing it on its platform and forwarding it to authorities, if they are available to collaborate in the reported area.\\
		SafeStreets also provides a feature that allows to see statistics about violations. In this way, for example, it is possible to know the most dangerous area in a city.\\
		SafeStreets can also collect information about incidents from authorities and combine them with the ones about traffic violations to let users do data-mining and to show them more accurate statistics. This advanced feature is only available in areas where authorities collaborate with SafeStreets.\\
		This last feature is especially productive for authorities, because SafeStreets can combine the retrieved information to suggest them some possible solutions for particularly unsafe areas.\\
	
		\subsection{Scope}
			A user can report a violation using SafeStreets application.
			\subsubsection{World and shared phenomena}
				In order to draw up a list of requirements, it is necessary to identify those events that occurs in the world. Some of them are shared with a machine: these phenomena define the interface through which machines interact with the world.\\\\				
				\textbf{World phenomena}
				\begin{itemize}
					\item Some people may not respect road traffic rules.
					\item Some people may park their own vehicle in a forbidden area.
					\item Accidents can occur.
				\end{itemize}
				\textbf{Shared phenomena}
				\begin{itemize}
					\item Authorities could generate traffic tickets thanks to reports. 
					\item A municipality could collect some suggestions, derived from data analysis.
					\item Traffic violations which are reported can be shown in a map.
				\end{itemize}
			\subsubsection{Goals}
				SafeStreets wants to provide to its users a service which aims to reach these goals:
				\begin{itemize}
					\goal{1}An individual can report traffic violations to SafeStreets.
					\goal{2}An individual can mine information about violations.					
					\goal{3}An authority can share its information about accidents occurred on its operative area with SafeStreets.
					\goal{4}An authority can retrieve suggestions of possible interventions from SafeStreets.
					\goal{5}An authority can generate traffic tickets through SafeStreets reports.
				\end{itemize}
				
				
		\subsection{Definitions, Acronyms, Abbreviations}
			\subsubsection{Definitions}
				\begin{itemize}
					\item \textbf{Operative area}: the geographic area in which an authority retains  control and actually works.
					\item \textbf{Registered authority}: an authority that decided to participate in SafeStreets initiative and that installed in its head office a particular version of the application that provides some features not available to common users (for example, the possibility to rate the traffic violation reports collected by SafeStreets).
					\item \textbf{Report image}: it is the image acquired through the SafeStreets app. It is obligatorily associated with a valid report.
					\item \textbf{Report timeout}: it is the timeout started when Safestreets' app acquires the report image. If the user doesn't complete the completion of the report by the deadline of this timeout, the report is considered invalid.
					\item \textbf{SafeStreets}: it is the crowdfunding app subject of this document.
					\item \textbf{Traffic violation report}: it is the message that SafeStreets collects through its app from users who want to report an alleged violation. It is often abbreviated as "report".
				\end{itemize}
			\subsubsection{Acronyms}
			 \textbf{TODO}
			\subsubsection{Abbreviations}
				\begin{itemize}
					\item {[Gn]}: n\textsuperscript{th} goal.
					\item {[Rn]}: n\textsuperscript{th} functional requirement.
					\item {[An]}: n\textsuperscript{th} domain assumption.
					\item {[UCn]}: n\textsuperscript{th} use case.
				\end{itemize}
		\subsection{Revision history}
			\begin{table}[h]
				\centering
				\begin{tabular}{c c c}
					\hline
					\textbf{Version} & \textbf{Last update} & \textbf{Comments} \\ 
					\hline
					1.0 &  \nth{10} November, 2019  & \\
					\hline
				\end{tabular}
				\caption{Revision history}
				\label{fig:Revision history}
			\end{table}
		
		\subsection{Reference Documents}
			\begin{itemize}
				\item "2019-2020 Software Engineering 2 mandatory project: goal, schedules and rules".
			\end{itemize}
			
		\subsection{Document Structure}
			In this part is shown how the document has been divided. For each chapter, is given a short description:
			\begin{itemize}
				\item \textit{Chapter 1} introduces the problem and describes an analysis of the world and of the shared phenomena. Here are included the purpose of SafeStreets application, its goals and some premises are formalized to provide a better understanding of following chapters.
				\item \textit{Chapter 2} includes further details on the shared phenomena and domain model, specifying which are the most important requirements and who will be a user of the system. Moreover, a formalization of assumptions and constraints is given.
				\item \textit{Chapter 3} analyzes more deeply all aspects of Chapter 2, providing more details for the development team: in this part, the focus is on requirements.
				\item \textit{Chapter 4} presents a formal analysis of the problem, using Alloy. Some worlds are generated from the model and assertions are checked.
				\item \textit{Chapter 5} shows the effort spent by each group member for this project.
				\item \textit{Chapter 6} simply includes the reference documents.
			\end{itemize}
	
	\newpage










	\section{Overall description}
		\subsection{Product perspective}
			Figure \ref{fig:domain_model} gives a visual representation of shared phenomena and domain model. 
			\begin{figure}[ht]
				\centering
				\includegraphics {diagrams/domain_model.png}
				\caption[Domain Model]{Domain Model}
				\label{fig:domain_model}
			\end{figure}
			
			The users can report traffic violations through the SafeStreets application.
			SafeStreets does not collect any data about users, so they are represented as a black box. The interaction between users and SafeStreets is established only through a report, which is published in the SafeStreets system.\\\\
			A registered authority can be noticed about traffic violations reported in its own operative area and eventually evaluate if that traffic violation report is valid or not. Moreover, an authority can report to SafeStreets any information about accidents in order to identify unsafe areas and suggest possible interventions.\\\\
			To understand the main events in SafeStreets system, some statechart diagrams are represented in figures below. \\\\
			\clearpage
			Figure \ref{fig:statechart_userReporting} explains how SafeStreets works when a user try to report a traffic violation.
			
			\begin{figure}[H]
				\centering
				\includegraphics[width=0.5\textwidth]{diagrams/statechart_UserSS.png}
				\caption[Statechart Diagram1]{Statechart Diagram about user report.}
				\label{fig:statechart_userReporting}
			\end{figure}
			
			The core of the application is based on user reports: \hl{the more reports there are, the more the system can provide information about areas in a municipality. Before publishing, SafeStreets does an input valuation: it doesn't have means to check if a report could actually be a traffic violations, but uses them to show areas from which reports are from.}
			
			\begin{figure}[H]
				\centering
				\includegraphics {diagrams/statechart_AuthoritySS.png}
				\caption[Statechart Diagram2]{Statechart Diagram about SafeStreets suggestions for an authority.}
				\label{fig:statechart_SuggestionsForAuthority}
			\end{figure}
			
			Figure \ref{fig:statechart_SuggestionsForAuthority} shows how SafeStreets operates with an authority.\\
			Authorities could eventually have some information about accidents in their operative zone and can decide to share them with SafeStreets.\\
			SafeStreets collects these data and try to cross them with reports about traffic violations: if for an area there are frequent violations or accidents, SafeStreets suggests possible interventions.
			
			\begin{figure}[H]
				\centering
				\includegraphics {diagrams/statechart_trafficTicket.png}
				\caption[Statechart Diagram3]{Statechart Diagram about SafeStreets system for traffic tickets generation.}
				\label{fig:statechart_trafficTickets}
			\end{figure}
			
			Finally, Figure \ref{fig:statechart_trafficTickets} \hl{represents SafeStreets system that allows an authority to show all violations reported in its operative area. Analysing a report, an authority can establish if it is actually valid in order to generate a traffic ticket for the offender. Otherwise, if report is not valid, the authority can delete it.}
			
			\clearpage	   
	
		\subsection{Product functions}
			In the following section all the main product functions are defined. The reporting and data-mining functions are considered the main ones, while the other are advanced ones offered by SafeStreets thanks to the collaboration with authorities. Thus, if authorities, in a defined area, are not sharing information with SafeStreets, these advanced functions are not available in that area.
			\subsubsection{Violation reporting}
				This is the core function of SafeStreets. The aim is to give to the user the possibility to report a traffic violation. The system allows the user to use this function without any previous login. The user is asked to complete his reporting action defining the following fields:\\
				- Picture representing the violation;\\
				- Type of violation.\\
				Other important information such as position, date and time of the violation are automatically detected by the application.\\
				In particular, to take the requested picture, the user is allowed only to use the in-app dedicated camera tool in order to prevent users from reporting past violations uploading old pictures and to avoid picture manipulations.\\ The picture is approved by the app if and only if the license plate is correctly recognized by the dedicated algorithm, otherwise, the app will ask to the user to take it again. \hl{If, after some attempts, SafeStreets app warns that is still not possible to recognize the license plate, the user is asked to insert it manually}. \\The type of violation is selectable only by the ones offered by SafeStreets.\\ The user is asked also to complete these two requests in a defined interval of time in order to guarantee the consistency of the report offered to SafeStreets.\\
				When a violation is correctly filled in, it is sent to the system which will publish it and will make it visible to all users.\\
				If an authority that covers the zone in which the violation occured is registered to SafeStreets, then the report is sent to it for evaluation.
		
		\subsubsection{Data-mining}
			This function allows the user to analyze SafeStreets data in order to mine some information, for example, by highlighting the streets (or the areas) with the highest frequency of violations, or the vehicles that commit the most violations. The kind of information requested can be filtered by user in order to restrict the research area. \\
		
		\subsubsection{Interactive suggestions}
			This function is strictly related to the collaboration between SafeStreets data, offered by users' reports, and authorities, which can enrich the database including information about the accidents that occur. \\SafeStreets can cross all these information and it can also identify potentially unsafe areas suggesting possible interventions to authorities. \\
			In a more detailed way, this function, analyzing data and making a comparison between them and some defined limit parameters, can automatically detect possible standard-defined improvements, create suggestions useful to reduce violations and accidents.
		
		\subsection{User characteristics}
			The actors of the application are the following:
			\begin{itemize}
				\item User: a person that is using SafeStreets app. He is a "guest" because he is not logged in, so no specific information about him are given. He can actively use the first two basic functions so he can send a violation reporting or do data-mining.
				\item Authority: special kind of user. It is allowed to use the user functions but it has also additional privileges on them such as to visualize more detailed information about reports. In particular, it is a registered identity that is responsible to manage its competent reports. \\An authority can send information about accidents of its area of competence to SafeStreets and can use SafeStreets statistics to identify the most egregious offenders.\\
		        A kind of authority is the Police, that can use reports to make traffic tickets validating SafeStreets reports.
			\end{itemize}
		
		\subsection{Assumptions, dependencies and constraints}
			This section specifies the assumptions made in the rest of the document, that is those facts that have been considered true a priori and that are not manageable by SafeStreets.
			\begin{itemize}
				\assumption{1} The device on which the app is installed has internet access.
				\assumption{2} The device on which the app is installed has geolocation features.
				\assumption{3} The device on which the app is installed has an external camera.
				\assumption{4} The device on which the app is installed is able to detect its position with a maximum error of five meters.
				\assumption{5} The position detected by the app coincides with the actual position detected by the geolocalizer.
				\assumption{6} The algorithm used by SafeStreets for reading a license plate from an image reads the correct number plate, if it is present in the image.
				\assumption{7} The SafeStreets software installed in the offices of the authorities registered with the service can only be used by the authorities themselves: it is not possible for an unauthorized user to gain access to such software.
				\assumption{8} Each position on earth can be associated with a unique postal code.
			\end{itemize}
	
	\section{Specific requirements}
		\subsection{External Interfaces Requirements}
			\subsubsection{User Interfaces}
			\subsubsection{Hardware Interfaces}
			\subsubsection{Software Interfaces}
			\subsubsection{Communication Interfaces}
		\subsection{Functional Requirements}
		\subsection{Performance Requirements}
		\subsection{Design Constraints}
			\subsubsection{Standards compliance}
			\subsubsection{Hardware limitations}
			\subsubsection{Any other constraint}
		\subsection{Software System Attributes}
			\subsubsection{Reliability}
			\subsubsection{Availability}
			\subsubsection{Security}
			\subsubsection{Maintainability}
			\subsubsection{Portability}
	
	\section{Formal analysis using Alloy}
	
	\section{Effort spent}
	
	\section{References}
	
	
\end{document}
